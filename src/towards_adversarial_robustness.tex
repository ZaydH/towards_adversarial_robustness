\documentclass[11pt,dvipsnames,usenames,aspectratio=169]{beamer}  % Add handout to options to disable overlays

% For more themes, color themes and font themes, see:
% http://deic.uab.es/~iblanes/beamer_gallery/index_by_theme.html
%
\mode<presentation>
{%
  \usetheme{CambridgeUS}    % or try default, Darmstadt, Warsaw, ...
  \usecolortheme{whale}     % or try albatross, beaver, crane, ...
  \usefonttheme{serif}          % or try default, structurebold, ...
  % \usefonttheme[onlymath]{serif}
  % \setbeamertemplate{navigation symbols}{}
  % \setbeamercovered{transparent}

  \setbeamercolor{title}{fg=white}
  \setbeamerfont{title}{series=\bfseries}
  \setbeamercolor{frametitle}{fg=black}
  \setbeamerfont{frametitle}{series=\bfseries}

  \setbeamercolor{section in head/foot}{fg=white}
  \setbeamerfont{section in head/foot}{series=\bfseries}
  \setbeamercolor{subsection in head/foot}{fg=white}
  \setbeamerfont{subsection in head/foot}{series=\bfseries}
  \setbeamercolor{author in head/foot}{fg=white}
  \setbeamerfont{author in head/foot}{series=\bfseries}
  \setbeamercolor{title in head/foot}{fg=white}
  \setbeamerfont{title in head/foot}{series=\bfseries}

  \setbeamercolor{block title}{use=structure,fg=white,bg=title in head/foot.bg}
  \setbeamerfont{block title}{series=\bfseries}
  \setbeamercolor{block body}{use=structure,fg=black,bg=black!1!white}
}

% Support graying out frame elements
\newcommand{\FrameOpague}{\setbeamercovered{again covered={\opaqueness<1->{40}}}}
% Transition slide
\newcommand{\transitionFrame}[1]{%
{%
  \begin{frame}[plain,noframenumbering]{}{} % the plain option removes the sidebar and header from the title page
    \setbeamertemplate{final page}[text]{\Large \textbf{#1}}
    \usebeamertemplate{final page}
  \end{frame}}
}

% \usepackage{hyperref}     % Loaded automatically by beamer
\usepackage{pgfplots}       % Used to generate embedded plots
\pgfplotsset{compat=1.13}

% Here's where the presentation starts, with the info for the title slide
\title[Towards Deep Robustness]{Towards Deep Learning Models \\ Resistant to Adversarial Attacks\texorpdfstring{~\cite{Madry:2017}}{}}
\author[\madry]{%
  \href{mailto:madry@mit.edu}{Aleksander Madry}\inst{1}  % \textsuperscript{(\Letter)}
  \and
  \href{mailto:amakelov@mit.edu}{Aleksandar Makelov}\inst{1}  % \textsuperscript{(\Letter)}
  \and
  \href{mailto:ludwigs@mit.edu}{Ludwig Schmidt}\inst{1}  % \textsuperscript{(\Letter)}
  \and
  \href{mailto:tsipras@mit.edu}{Dimitris Tsipras}\inst{1}  % \textsuperscript{(\Letter)}
  \and
  \href{mailto:avladu@mit.edu}{Adrian Vladu}\inst{1}  % \textsuperscript{(\Letter)}
  % \href{mailto:lowd@cs.uoregon.edu}{Daniel Lowd}\inst{1}
}

\institute[MIT]{%
  \textsuperscript{1}\textbf{MIT -- CSAIL}\\
  % \texttt{{zayd, lowd}@ucsc.edu}
}
\date{October~18,~2019}

\newcommand{\etal}{et~al.}
\newcommand{\elkan}{Elkan \&~Noto}

% Used for including standalone docs
\usepackage{standalone}

% Imported via UltiSnips
% Unbreakable dash:
%  Words hyphened with these dashes can also be broken at other positions than the dash
%    \-/ hyphen
%    \-- en-dash
%    \--- em-dash
%    extdash unbreakable dashes
%
%  No line breaks possible at the hyphen
%    \=/ hyphen
%    \== en-dash
%    \=== em-dash
\usepackage[shortcuts]{extdash}

% Imported via UltiSnips
\usepackage{color}
\newcommand{\colortext}[2]{{\color{#1} #2}}
\newcommand{\red}[1]{\colortext{red}{#1}}
\newcommand{\blue}[1]{\colortext{red}{#1}}
\newcommand{\green}[1]{\colortext{green}{#1}}

% Imported via UltiSnips
\usepackage{amsmath}
\DeclareMathOperator*{\argmax}{arg\,max}
\DeclareMathOperator*{\argmin}{arg\,min}
\DeclareMathOperator{\sgn}{sgn}
\usepackage{amsfonts}  % Used for \mathbb and \mathcal
\usepackage{amssymb}

% Imported via UltiSnips
\usepackage{mathtools} % for "\DeclarePairedDelimiter" macro
% \swapifbranches changes unstarred paired delimiters to starred and
% vice versa.  This means by default, paired delimiters have the star.
\usepackage{etoolbox}
\newcommand\swapifbranches[3]{#1{#3}{#2}}
\makeatletter
\MHInternalSyntaxOn
\patchcmd{\DeclarePairedDelimiter}{\@ifstar}{\swapifbranches\@ifstar}{}{}
\MHInternalSyntaxOff
\makeatother
% Place after swap to ensure swap star
\DeclarePairedDelimiter{\sbrack}{\lbrack}{\rbrack}
\DeclarePairedDelimiter{\floor}{\lfloor}{\rfloor}
\DeclarePairedDelimiter{\ceil}{\lceil}{\rceil}
\DeclarePairedDelimiter{\abs}{\lvert}{\rvert}
\DeclarePairedDelimiter{\norm}{\lVert}{\rVert}
\usepackage{bm}
\DeclarePairedDelimiterX\set[1]\lbrace\rbrace{#1}
\DeclarePairedDelimiterX\setbuild[2]\lbrace\rbrace{#1 \bm: #2}
\newcommand{\ints}[1]{{\sbrack{#1}}}
\newcommand{\func}[3]{{#1:#2\rightarrow#3}}
\newcommand{\defeq}{\stackrel{\mathclap{\mbox{\tiny def}}}{=}}

% Imported via UltiSnips
\usepackage{multirow}
\usepackage{booktabs}

% Imported via UltiSnips
\usepackage{tikz}
\usetikzlibrary{arrows,decorations.markings,shadows,positioning,calc,backgrounds,shapes}

\usepackage{graphicx}
\graphicspath{ {./img/} }

\newcommand{\distr}{\mathcal{D}}
\newcommand{\X}{x}
\newcommand{\y}{y}

\newcommand{\loss}{L}
\newcommand{\params}{\theta}

\newcommand{\sPerturb}{\mathcal{S}}

\renewcommand{\green}[1]{{\color{ForestGreen} #1}}


\begin{document}

\begin{frame}
  \titlepage
\end{frame}

\section{Introduction}
\begin{frame}{Why is Adversarial Robustness Important?}
  \onslide<+->{Machine learning-based classifiers are increasingly the center of \green{security-critical systems}.}
  \begin{itemize}[<+->]
    \item \textit{Examples}: Autonomous cars, facial recognition, \& malware detection
  \end{itemize}
  \vfill
  \onslide<+->{Classifiers can very effectively classify \green{benign} inputs}
  \begin{itemize}[<+->]
    \setlength\itemsep{12pt}
    \item \textit{Very small} changes to an input can fool state-of-the-art classifiers with \textit{high confidence}
    \item \textbf{Crucial Design Goal}: Resistance to \textit{adversarially-chosen inputs}
  \end{itemize}
\end{frame}

\section{Background}
\begin{frame}{What is an Adversarial Example?}
  \onslide<+->{
    \begin{definition}
      \textit{\green{Adversarial Example}}: An input chosen/modified by an adversary that is almost indistinguishable from natural data but is (confidently) misclassified by the network.
    \end{definition}
  }

  \begin{columns}
    \begin{column}{0.23\textwidth}
      \onslide<+->{
        \begin{center}
          \includegraphics{adv_panda}

          \textbf{\blue{Prediction}}
          \\
          \textbf{Panda}
          \\
          55.7\% Confidence~\cite{Goodfellow:2014}
        \end{center}
      }
    \end{column}
    \onslide<+->{
      \begin{column}{0.05\textwidth}
          \begin{center}
            +\vspace{1.6cm}
          \end{center}
      \end{column}
      \begin{column}{0.20\textwidth}
          \begin{center}
            \includegraphics{adv_noise}

            \textbf{\blue{Prediction}}
            \\
            \textbf{Nematode}
            \\
            8.2\% Confidence
          \end{center}
      \end{column}
    }
    \onslide<+->{
      \begin{column}{0.05\textwidth}
          \begin{center}
            =\vspace{1.6cm}
          \end{center}
      \end{column}
      \begin{column}{0.20\textwidth}
          \begin{center}
            \includegraphics{adv_combined}
            \textbf{\blue{Prediction}}
            \\
            \textbf{Gibbon}
            \\
            99.3\% Confidence
          \end{center}
      \end{column}
    }
    \onslide<+->{
      \begin{column}{0.20\textwidth}
          \begin{center}
            \includegraphics[scale=0.15]{gibbon}
            \textbf{\red{Actual}\\Gibbon}
          \end{center}
      \end{column}
    }
  \end{columns}

\end{frame}

\begin{frame}{$\ell_{p}$ Balls --- Norms First}
  For ${x \in \mathbb{d}}$, the $L_{p}$ norm is:

  \begin{equation}\label{eq:LpNorm}
    \norm{x}_{p} = \left( \sum_{i=1} x_{i}^{p}  \right)^{\frac{1}{p}}
  \end{equation}

  $L_{\infty}$ norm is a special a case:

  \begin{equation}\label{eq:LinftyNorm}
    \norm{x}_{\infty} = \sup_{i} \abs{x_i}
  \end{equation}

  \begin{center}
    \textbf{Note}: $\sup$ equals the $\max$ for a finite set
  \end{center}
\end{frame}

\begin{frame}{$\ell_{p}$ Balls --- Let's Visualize\ldots}
  \begin{definition}
    Given scalar ${\varepsilon > 0}$, the $\ell_{p}$ ball of a point ${x \in \mathbb{R}^{d}}$ is:

    \begin{equation}\label{eq:LpBall}
      \ell_{p}(x) = \setbuild{x + \delta}{\norm{\delta}_{p} \leq \varepsilon}\text{.}
    \end{equation}
  \end{definition}

  \onslide<+->{
    \begin{center}
      \includegraphics[scale=0.29]{img/lpballs.pdf} \cite{wiki:Lp_space}
    \end{center}
  }
\end{frame}


\begin{frame}{Attack Paradigms}
  \madry\ study adversarial robustness under two different attack paradigms:
  \vfill
  \begin{enumerate}[<+->]
    \item \textbf{Black-Box}: Adversary has no direct access to target network
      \begin{itemize}[<+->]
        \setlength\itemsep{6pt}
        \item \red{Weaker} attack paradigm
        \item Adversary may have \textit{rough} information, e.g.,~model architecture \& training dataset
          \begin{itemize}
            \item \textit{Example}: Transfer attack
          \end{itemize}
        \item Generally \textit{one-shot} attacks
      \end{itemize}
    \vfill
    \item \textbf{White-Box}: Attacker has access to target network's parameter
      \begin{itemize}
        \setlength\itemsep{6pt}
        \item \green{Stronger} attack paradigm
        \item \textit{Examples}: PGD and FGSM (both discussed in this talk)
        \item Enables \textit{iterative} attacks, e.g.,~refined probing
      \end{itemize}
  \end{enumerate}
\end{frame}


\begin{frame}{Nomenclature}
  \onslide<+->{Quite standard and used throughout this talk.}  \onslide<+->{\textit{If you have a question,} \textbf{ask now!}}

  \begin{itemize}[<+->]
    \setlength{\itemsep}{6pt}
    \item ${\X \in \domainX}$: feature values
    \item ${\y \in \domainY}$: Ground truth (true) label
    \item $\func{\distr}{\domainX \times \domainY}{\mathbb{R}_{{\geq}0}}$: Underlying sample data probability distribution s.t.\ ${(\X,\y) \sim \distr}$

    \vspace{6pt}
    \item ${\params \in \domainP}$: Model (neural network) parameters

    \vspace{6pt}
    \item ${\sPerturb \subseteq \domainX}$: Set of allowed (adversarial) perturbations can be applied to any~$\X$
    \item ${\perturb \in \sPerturb}$: (Adversarial) perturbation s.t.
      \begin{equation}\label{eq:AdversarialX}
        \X^{\text{adv}} = x + \delta
      \end{equation}

    \vspace{6pt}
    \item $\mathbb{E}_{x \in \mathcal{X}}\sbrack{f(x)}$: Expected value (weighted mean) of $f(x)$ for all ${x \in \mathcal{X}}$
  \end{itemize}
\end{frame}


\transitionFrame{\textbf{Contribution \#1}: Define a \blue{tractable} learner that is \green{adversarially-robust} against a universal \textit{first-order adversary}}

\begin{frame}{What is a ``first-order adversary''?}
  \onslide<+->{
    \begin{definition}{First-order Adversary}
      The strongest attack (e.g.,~white-box) utilizing only \textit{first-order} information about the network
    \end{definition}
  }

\end{frame}

\begin{frame}{``Standard'' Classification Model}
  \begin{itemize}[<+->]
    \item Neural network training is based on \textit{\blue{empirical risk minimization}} (ERM)
    \item \textbf{\green{Basic Goal}}
      \begin{equation}\label{eq:ERM}
        \min_{\params} \mathbb{E}_{(\X,\y) \sim \distr} \sbrack{\loss \left( \X, \y ; \params \right)}
      \end{equation}

    \item \textit{\blue{Attack Model}}: For any ${x \in \domainX}$, adversary can make a set of allowed perturbations~$\sPerturb$
      \begin{itemize}
        \item \textit{Theoretical Paradigm}: $\sPerturb = \linf\text{-ball}$
      \end{itemize}

    \item \textbf{\red{Big Problem}}: ERM-trained models are not robust to adversarially perturbed examples~\cite{Biggio:2013,Szegedy:2013}
  \end{itemize}
\end{frame}

\subsection{Minimax}
\begin{frame}{Minimax Framework}
  \begin{definition}
    \blue{\textit{Minimax optimization}} reformulates ERM as:
    \onslide<2->{
      \begin{equation}\label{eq:Minimax}
        \textcolor<5->{ForestGreen}{\min_{\params} \rho(\params)}\text{, where } \rho(\params) = \mathbb{E}_{(\X,\y) \sim \distr} \sbrack{\textcolor<4->{red}{ \max_{\delta \in \sPerturb} \loss (\X + \perturb, \y ; \params)}} \text{.}
      \end{equation}
    }
  \end{definition}

  \vspace{8pt}
  \only<3-6>{Two parts to minimax optimization framework
    \begin{itemize}
      \setlength{\itemsep}{4pt}
      \item \onslide<4->{{\color{red}Inner Maximization}: Find \red{worst case} ${\xadv = \X + \perturb}$ with highest loss}
      \item \onslide<5->{{\color{ForestGreen}Outer Minimization}: Find model parameters~$\params$ that minimize adversarial loss}
    \end{itemize}
  }
  \only<7->{
    \textbf{Question}: What can we say if ${\textcolor{ForestGreen}{\min_{\params} \rho(\params)} \rightarrow 0}$?

    \vspace{4pt}
    \onslide<8->{\textbf{Answer}: \green{Perfect robustness}. \textit{Guarantee} no perturbation defined in attack model ($\sPerturb$) fools the network.}

    \vspace{10pt}
    \onslide<9->{\textbf{Why?}} \onslide<10->{Inner maximization considers \red{worst case} adversary so guarantees all cases}
   }
\end{frame}


\begin{frame}{So...Are We Done Here?}

\end{frame}


\begin{frame}{Two Primary Questions}

\end{frame}

\section{Producing Strong Adversarial Examples}
\transitionFrame{Producing Strong Adversarial Examples}

\begin{frame}{Adversarial Examples should be Intractable}

  \onslide<+->{\textbf{\green{Inner Maximization}}: Adversarial example generation problem:
      \begin{equation}\label{eq:InnerMaximization}
        \max_{\delta \in \sPerturb} \loss (\X + \perturb, \y ; \params)
      \end{equation}

    \vspace{-3pt}
    is a highly non-concave function.  Potentially large number of local maxima making this problem \textit{seem} intractable.
  }

  \vfill
  \begin{itemize}[<+->]
    \setlength{\itemsep}{8pt}
    \item \textbf{\red{Previous Work}}: Fast Gradient Sign Method (FGSM)
    \item \textbf{\blue{\madry's Solution}}: Projected Gradient Descent (PGD)
  \end{itemize}
\end{frame}


\begin{frame}{Previous Work: Fast Gradient Sign Method}
  \onslide<+->{%
    \begin{definition}
      \textbf{\blue{Fast Gradient Sign Method}} (FGSM) linearizes the inner maximization s.t.:

      \begin{equation}\label{eq:FGSM}
        \xadv = \X + \varepsilon \sgn\left(\nabla_{\X} \loss(\X, \y ; \params) \right)
      \end{equation}
    \end{definition}
  }

  \begin{itemize}[<+->]
    \setlength\itemsep{8pt}
    \item \textbf{Summary}: Simple ``single-step-size'' approach
    \item Enhanced versions of FGSM (e.g., multi-step, randomized) that are beyond the scope of this talk
  \end{itemize}

  \vspace{3pt}
  \onslide<+->{\textbf{\red{Major Issue}}: FGSM provides no security guarantee}
  \begin{itemize}[<+->]
    \setlength\itemsep{8pt}
    \item \textbf{Why?} \onslide<+->{Not selecting the \textit{worst-case} perturbation}
    \item Easy to find ``nearby'' perturbations with significantly higher loss
  \end{itemize}
\end{frame}


\begin{frame}{Projected Gradient Descent}
  \onslide<+->{A type of \green{\textit{constrained optimization}} with the constraint is a \blue{\textit{projection}}}

  \vfill
  \onslide<+->{
    \begin{definition}
      A \textbf{\blue{projection}} of a point~$z$ onto a set~$\mathcal{X}$ is defined as:

      \[ \Pi_{\mathcal{X}}(z) = \argmin_{\X \in \mathcal{X}} \norm{\X-z}^{p}\text{.}  \]
    \end{definition}
  }

  \vfill
  \begin{itemize}[<+->]
    \item $\Pi_{X}$ is the projection operator
  \end{itemize}
\end{frame}


\begin{frame}{Projected Gradient Descent --- The Algorithm}
  \begin{columns}
    \begin{column}{0.5\textwidth}
      \textbf{Procedure}:
      \begin{enumerate}[<+->]
        \setlength{\itemsep}{12pt}
        \item For any ${\X \in \mathcal{X}}$, define ${\X^{(0)} = \X + \delta}$ where ${\perturb \sim \mathcal{U}\left( \sPerturb \right)}$
        \item For each time $t$, define:
          \begin{equation}\label{eq:PGD:GradUpdate}
            z^{(t+1)} = \X^{(t)} \textcolor<+->{red}{-} \alpha \nabla_{\X} \loss(\X^{(t+1)}, \y_{\X}) \text{.}
          \end{equation}
        \item Enforce the constraint where:
          \begin{equation}\label{eq:PGD:NextX}
            \X^{(t+1)} = \Pi_{\mathcal{\mathcal{X}}}(z^{(k+1)}) \text{.}
          \end{equation}

        \item Repeat steps~\#2 and~\#3 until convergence
      \end{enumerate}
    \end{column}
    \begin{column}{0.45\textwidth}
      \begin{center}
        \onslide<+->{\includegraphics[scale=0.65]{pgd}~\cite{Srebro}}
      \end{center}
    \end{column}
  \end{columns}
\end{frame}

\subsection{\texorpdfstring{$\ell_{p}$}{Lp} Balls}
\transitionFrame{To go further, we need to understand $\ell_{p}$ balls\ldots}

\begin{frame}{$\ell_{p}$ Balls --- Norms First}
  For ${x \in \mathbb{d}}$, the $L_{p}$ norm is:

  \begin{equation}\label{eq:LpNorm}
    \norm{x}_{p} = \left( \sum_{i=1} x_{i}^{p}  \right)^{\frac{1}{p}}
  \end{equation}

  $L_{\infty}$ norm is a special a case:

  \begin{equation}\label{eq:LinftyNorm}
    \norm{x}_{\infty} = \sup_{i} \abs{x_i}
  \end{equation}

  \begin{center}
    \textbf{Note}: $\sup$ equals the $\max$ for a finite set
  \end{center}
\end{frame}

\begin{frame}{$\ell_{p}$ Balls --- Formally}
  \begin{definition}
    Given scalar ${\varepsilon > 0}$, the $\ell_{p}$~ball of a point ${x \in \mathbb{R}^{d}}$ is:

    \begin{equation}\label{eq:LpBall}
      \ell_{p}(x) = \setbuild{x + \delta}{\norm{\delta}_{p} \leq \varepsilon}\text{.}
    \end{equation}
  \end{definition}

  \begin{columns}
    \begin{column}{0.5\textwidth}
      \onslide<2->{Let's visualize $\ell_{p}$ for different values of $p$}

      \vspace{15pt}
      \onslide<4->{\blue{\textbf{Question}}: What is the value of $\varepsilon$?}

      \vspace{4pt}
      \onslide<5->{\textbf{Answer}: $\varepsilon = 1$}

      \vspace{15pt}
      \onslide<6->{\green{\textbf{Key Takeaway}}: $\ell_{\infty}$~ball is a superset of all other $\ell_{p}$~balls for fixed $\varepsilon$}
    \end{column}
    \begin{column}{0.45\textwidth}
      \onslide<3->{
        \begin{center}
          \includegraphics[scale=0.28]{img/lpballs.pdf}~\cite{wiki:Lp_space}
        \end{center}
      }
    \end{column}
  \end{columns}
\end{frame}


\subsection{\texorpdfstring{$\ell_{\infty}$}{L-infinity}~Ball \& PGD}

\begin{frame}{Connecting $\ell_{\infty}$ \& PGD}

\end{frame}

\begin{frame}{PGD \& Adversarial Tractability}
  \onslide<+->{\textbf{\blue{Question}}: Does PGD + $\ell_{\infty}$~balls guarantee eliminate \textit{all} adversarial examples within (Minkowski) distance~$\varepsilon$ of $\X$?}

  \vspace{3pt}
  \onslide<3->{\textbf{Answer}: No.} \only<4-5>{Why?}
  \begin{itemize}
    \item \onslide<5->{PGD is only a \textbf{first-order adversary}.}
  \end{itemize}

  \vfill
  \onslide<6->{
    \begin{definition}
      \blue{\textbf{First-order Adversary}} is the strongest attack utilizing only \textit{first-order} (e.g.,~gradient) information about network
    \end{definition}
  }

  \vfill
  \onslide<7->{An attacker using higher order information is used (e.g.,~Hessian) may tractably find adversarial examples.}
  \begin{itemize}
    \item \onslide<8->{Similar to a \textit{polynomially bounded} adversary that is the cornerstone of cryptography}
  \end{itemize}
\end{frame}

\subsection{Zero-Order Adversaries}
\begin{frame}{Zero-Order Adversaries}

\end{frame}


\section{Adversarial Example Landscape}
\transitionFrame{Adversarial Example Landscape}

\begin{frame}{Investigating the ``Inner Maximization''}
  \onslide<+->{%
    \begin{equation}
      \min_{\params} \rho(\params) \text{, where } \rho(\params) = \mathbb{E}_{(\X,\y) \sim \distr} \sbrack{\red{\max_{\delta \in \sPerturb} \loss (\X + \perturb, \y ; \params)}}
    \end{equation}
  }

  \begin{itemize}[<+->]
    \setlength{\itemsep}{20pt}
    \item Preceding theoretical analysis of \textbf{\red{inner maximization}} described PGD's usefulness to provide guarantees regarding first-order adversaries
    \item \textbf{Goal of this Section}: Demonstrate \textit{empirically} that the theoretical analysis holds even in environments that are theoretically \textit{intractable}
      \begin{itemize}[<+->]
        \setlength{\itemsep}{8pt}
        \item \textbf{Recall}: Inner maximization is highly \textit{non-concave} and not continuously-differentiable
        \item \textbf{Question}: Why are we interested in concavity and not convexity?
      \end{itemize}
  \end{itemize}
\end{frame}


\begin{frame}{Experiment Setup}
  \onslide<+->{Setup applies for all experiments in this section}
  \begin{itemize}
    \setlength{\itemsep}{10pt}
    \item \textbf{Datasets}: MNIST \& CIFAR10

    \item \textbf{Procedure}: For each iteration
      \begin{enumerate}[<+->]
        \setlength\itemsep{6pt}
        \item Select example,~$\X$, u.a.r.\ from the dataset
        \item Select perturbation,~${\perturb \in \sPerturb}$ u.a.r.
        \item Perform PGD on perturbed example, ${\X + \perturb}$
      \end{enumerate}

    \item \textbf{\# Random Restarts}: Varies by experiment

    \item \textbf{Loss Function}: Cross-entropy
      \begin{equation}\label{eq:CrossEntropy}
        \loss(y,\hat{y}) = \sum_{c \in \mathcal{C}} -y_c \log \left( \hat{y}_{c} \right)
      \end{equation}
  \end{itemize}
\end{frame}


\begin{frame}{Experiment~\#1: $\Delta\loss$ vs.\ \#Iterations}
  \onslide<+->{\textbf{Goal}: Study change in loss versus number of iterations of PGD}
  \begin{itemize}[<+->]
    \item \textbf{\# Random Restarts}: 20
  \end{itemize}

  \begin{columns}
    \begin{column}{0.23\textwidth}
      \begin{center}
        \onslide<+->{\includegraphics[scale=0.32]{loss_v_iter/mnist_standard.pdf}}
      \end{center}
    \end{column}
    \begin{column}{0.2\textwidth}
      \begin{center}
        \onslide<+->{\includegraphics[scale=0.32]{loss_v_iter/mnist_adv.pdf}}
      \end{center}
    \end{column}
    \begin{column}{0.21\textwidth}
      \begin{center}
        \onslide<+->{\includegraphics[scale=0.32]{loss_v_iter/cifar_standard.pdf}}
      \end{center}
    \end{column}
    \begin{column}{0.22\textwidth}
      % \vspace{-9pt}
      \begin{center}
        \onslide<+->{\includegraphics[scale=0.32]{loss_v_iter/cifar_adv.pdf}}
      \end{center}
    \end{column}
  \end{columns}
  \vfill
  \onslide<+->{\textbf{Takeaways}}
  \begin{itemize}[<+->]
    \item Adversarial training significantly reduces loss on adversarial examples.
    \item Loss values are \textbf{\blue{well-concentrated}}
      \begin{itemize}
        \item Echoes folklore belief neural network training possible since many local minima with similar loss values
      \end{itemize}
  \end{itemize}
\end{frame}


\begin{frame}{Experiment~\#2: Absence of Outliers}
  \onslide<+->{\textbf{Goal}: Verify security guarantee for a large number of examples \& iterations}
  \begin{itemize}[<+->]
    \item \textbf{\# Examples} ($x$): 5
    \item \textbf{\# Iterations}: 100K
  \end{itemize}

  \vspace{-15pt}
  \begin{columns}
    \begin{column}{0.7\textwidth}
      \begin{center}
        \onslide<4->{\includegraphics[scale=0.19]{loss_hist/mnist}}

        \onslide<6->{\includegraphics[scale=0.19]{loss_hist/cifar}}
      \end{center}
    \end{column}
    \begin{column}{0.25\textwidth}
      \vspace{20pt}
      \onslide<5->{
        \begin{itemize}
          \setlength{\itemsep}{20pt}
          \item \textbf{\blue{Blue}}: Standard training
          \item \textbf{\red{Red}}: Adversarial training
        \end{itemize}
      }
    \end{column}
  \end{columns}

  \vfill
  \onslide<7->{\green{\textbf{Takeaway}}: No outliers (i.e., high loss adversarial examples) \& concentrated losses}
\end{frame}


\begin{frame}{Experiment~\#3: Adversarial Example Mode Collapse}
  \onslide<+->{\textbf{\blue{Mode Collapse}}: Common problem in GANs where the generated output where the generated outputs have limited diversity.}
  \vfill
  \onslide<+->{\green{\textbf{Goal}}: Demonstrate that the generated adversarial examples are noticeably distinct:}
  \begin{itemize}[<+->]
    \item \# Iterations: 10,000
    \item \textbf{Metric}: Inter-adversarial example (Euclidean) distance
  \end{itemize}
  \vfill
  \onslide<+->{\textbf{\green{Result}}: Inter-maxima distance is distributed close to the expected distance between \textit{two random points} in the $\ell_{\infty}$\-/ball and cosine similarity between points is close to 90\textsuperscript{$\circ$}.}
  \begin{itemize}[<+->]
    \item Empirically demonstrates no adversarial example ``mode collapse''
    \item Recall the ``curse of dimensionality'' so take with a grain of salt
  \end{itemize}
\end{frame}


\section{Adversarial Training}

\transitionFrame{Adversarially Robust Training}

\begin{frame}{Solving the Outer Minimization}
  \onslide<+->{%
    \begin{equation}\label{eq:MinimaxTrainRepeat}
      \green{\min_{\params} \rho(\params)} \text{, where } \rho(\params) = \mathbb{E}_{(\X,\y) \sim \distr} \sbrack{\red{\max_{\delta \in \sPerturb} \loss (\X + \perturb, \y ; \params)}}
    \end{equation}
  }

  \begin{itemize}[<+->]
    \item We talked about solving the \red{inner maximization} to create the adversarial examples.

    \vspace{13pt}
    \item \textbf{Question}: What do we need to solve to train adversarially robust networks?
    \vspace{5pt}
    \item \textbf{Answer}: Solve the \green{outer minimization}

    \vspace{13pt}
    \item \textbf{Question}: What algorithm can use to solve the outer maximization?
    \vspace{5pt}
    \item \textbf{Answer}: Stochastic gradient descent (SGD) on the adversarial examples
      \begin{itemize}[<+->]
        \setlength{\itemsep}{6pt}
        \item \textit{Intuition}: SGD on adv.\ examples' loss reduces the \red{inner maximization}
        \item \textit{Takeaway}: Given an algorithm that transforms training examples into adv.\ examples (e.g.,~PGD), the rest of the \textbf{\blue{training process proceeds normally}}
      \end{itemize}
  \end{itemize}
\end{frame}


\begin{frame}{Why Should I Believe the Preceding Claims are True?}
  \onslide<+->{}
  \onslide<+->{\textbf{Answer}: You shouldn't.  The preceding explanation is definitely \red{not a proof}.}

  \vspace{20pt}
  \onslide<+->{\madry\ rely on \textbf{\blue{Danskin's Theorem}} that states gradients at inner maximizers correspond to descent directions for the complete problem}

  \vspace{20pt}
  \onslide<+->{\textbf{Problem}: Multiple assumptions made by Danskin's Theorem's \red{do not apply} here, e.g.,~continuously differentiable function, only approximate inner maximizers etc.}
  \begin{itemize}[<+->]
    \setlength{\itemsep}{6pt}
    \item Empirical results show that despite Danskin's not holding the algorithm is addressing the issues
    \item A more complete discussion of how Danksin's applies to this problem is in Appendix~A (see Arxiv version) and is beyond the scope of this talk
  \end{itemize}
\end{frame}


\section{Network Capacity \& Adversarial Robustness}
\transitionFrame{Network Capacity \& Adversarial Robustness}

\begin{frame}
	\onslide<+->{Solving the following equation is not enough to guarantee the robustness and accuracy of the classifier.
		\begin{equation*}
		{\min_{\params} \rho(\params)}\text{, where } \rho(\params) = \mathbb{E}_{(\X,\y) \sim \distr} \sbrack{{ \max_{\delta \in \sPerturb} \loss (\X + \perturb, \y ; \params)}}
		\end{equation*}
	}

	\onslide<+->{The problem must have a small \colortext{green}{\textit{value}}, which means that the final loss achieved by the classifier against adversarial examples must be small.}
	
	\vspace{4pt}
	
	\onslide<+->{Therefore, a very small \colortext{blue}{$\rho(\params)$} corresponds to a perfect classifier that is robust to adversarial inputs.}
	
	\vspace{4pt}
	
	\onslide<+->{For a fixed set $\perturb$, the value of the problem is entirely dependent on the
		architecture of the classifier.}
	
	\vspace{4pt}
	
	\onslide<+->{Therefore, the architectural capacity of the model is very important for its over performance.}
\end{frame}

\begin{frame}{Visualizing the Effect of Capacity}
	\begin{columns}
		\begin{column}{0.32\textwidth}
			\onslide<+->{
				\begin{center}
					\includegraphics[scale=0.3]{capacity/benign_only.pdf}
					
					(Benign) binary classification using a \\\blue{linear decision boundary}
				\end{center}
			}
		\end{column}
		\begin{column}{0.32\textwidth}
			\onslide<+->{
				\begin{center}
					\includegraphics[scale=0.3]{capacity/misclassified_adv.pdf}
					
					Adversarial $\ell_{\infty}$ balls
					\\
					\red{$\star$} misclassified \\ adversarial examples
				\end{center}
			}
		\end{column}
		\begin{column}{0.32\textwidth}
			\onslide<+->{
				\begin{center}
					\includegraphics[scale=0.3]{capacity/complicated_decision.pdf}
					
					``Complicated'' decision boundary required
				\end{center}
			}
		\end{column}
	\end{columns}
\end{frame}

\begin{frame}{}
	\begin{columns}
		\begin{column}{0.48\textwidth}
			\onslide<+->{
				\begin{center}
					\includegraphics[scale=0.18]{nc_ar/mnist.png}
					
					MNIST
				\end{center}
			}
		\end{column}
		\begin{column}{0.48\textwidth}
			\onslide<+->{
				\begin{center}
					\includegraphics[scale=0.44]{nc_ar/cifar10.png}
					
					CIFAR10
				\end{center}
			}
		\end{column}
	\end{columns}
\end{frame}

\begin{frame}
	\onslide<+->{Verified by their experiments, capacity is crucial for:
		\begin{itemize}[<+->]
			\setlength\itemsep{8pt}
			\item Robustness
			\item The ability to successfully train a classifier against strong adversaries. 
		\end{itemize}
	}
	\begin{columns}
		\begin{column}{0.99\textwidth}
			\onslide<+->{
				\begin{center}
					\includegraphics[scale=0.38]{nc_ar/fig4.png}
				\end{center}
			}
		\end{column}
	\end{columns}
\end{frame}

\begin{frame}
	\onslide<+->{What they found:
		\begin{itemize}[<+->]
			\setlength\itemsep{8pt}
			\item Capacity alone helps
			\item FGSM adversaries don’t increase robustness (for large  $\varepsilon$)
			\item Weak models may fail to learn non-trivial classifiers
			\item The value of the saddle point problem decreases as we increase the capacity
			\item More capacity and stronger adversaries decrease transferability
		\end{itemize}
	}
\end{frame}

\subsection{Experiments}
\begin{frame}
	\onslide<+->{Two key elements in training their robust classifiers
		\begin{itemize}[<+->]
			\setlength\itemsep{8pt}
			\item A sufficiently high capacity network
			\item The strongest possible adversary
		\end{itemize}
	}
	\vspace{15pt}
	\onslide<+->{Using a "complete" first-order adversary by PGD for both MNIST and CIFAR10:
		\begin{itemize}[<+->]
			\setlength\itemsep{8pt}
			\item Multiple epochs for training the model. Therefore, no benefit from restarting PGD multiple times per batch
		\end{itemize}
	}
\end{frame}


\begin{frame}
	The steady decrease in the training loss of adversarial examples indicates successfully solution for the original optimization problem during training.

	\begin{columns}
		\begin{column}{0.99\textwidth}
			\onslide<+->{
				\begin{center}
					\includegraphics[scale=0.3]{nc_ar/fig5.png}
				\end{center}
			}
		\end{column}
	\end{columns}
\end{frame}

\begin{frame}
	\onslide<+->{Models are trained against a range of adversaries:
		\begin{itemize}[<+->]
			\setlength\itemsep{8pt}
			\item White-box attacks with PGD for a different number of iterations and restarts (A)
			\item White-box attacks with PGD using the Carlini-Wagner (CW), and CW+ when ${\kappa = 50}$
			\item Black-box attacks from an independently trained copy of the network (A$^{'}$).
			\item Black-box attacks from a version of the same network trained only on natural examples (A$_{nat}$)
			\item Black-box attacks from a different convolution architecture (B)
		\end{itemize}
	}
\end{frame}

\begin{frame}
	\begin{columns}
		\begin{column}{0.99\textwidth}
			\onslide<+->{
				\begin{center}
					\includegraphics[scale=0.3]{nc_ar/table1.png}
				\end{center}
			}
		\end{column}
	\end{columns}
\end{frame}

\begin{frame}
	\begin{columns}
		\begin{column}{0.99\textwidth}
			\onslide<+->{
				\begin{center}
					\includegraphics[scale=0.3]{nc_ar/table2.png}
				\end{center}
			}
		\end{column}
	\end{columns}
\end{frame}

\begin{frame}
	\begin{columns}
		\begin{column}{0.99\textwidth}
			\onslide<+->{
				\begin{center}
					\includegraphics[scale=0.3]{nc_ar/fig6.png}
				\end{center}
			}
		\end{column}
	\end{columns}
\end{frame}

\begin{frame}
	\onslide<+->{Conclusion:
		\begin{itemize}[<+->]
			\setlength\itemsep{15pt}
			\item Although deep neural networks very vulnerable to adversarial attacks, they can be made resistant to adversarial attacks
			\item Based on the theory and experiments, reliable adversarial training methods can be designed because of the unexpectedly regular structure of the underlying optimization task:
			\begin{itemize}[<+->]
				\setlength\itemsep{15pt}
				\item Even though the relevant problem corresponds to the maximization of a highly
				non-concave function with many distinct local maxima, their values are highly concentrated.
			\end{itemize}
		\end{itemize}
	}
\end{frame}

\section{Related Works}
\begin{frame}
	\begin{itemize}[<+->]
		\setlength\itemsep{15pt}
		\item Robust optimization, not a new topic
		\item Adversarial ML, not a new topic
		\item Having the above in the context of DNN is their contribution
	\end{itemize}

	\onslide<+->{
		Recent work on adversarial training on ImageNet also observed that the model capacity is important for adversarial training. Their contribution
		\begin{itemize}[<+->]
			\setlength\itemsep{15pt}
			\item Training against multi-step methods (PGD) does lead to resistance against adversaries
		\end{itemize}
	}
\end{frame}

\begin{frame}
	\onslide<+->{
		There are previous works on the min-max optimization problem. But, there are three differences:
		\begin{itemize}[<+->]
			\setlength\itemsep{15pt}
			\item (Previous works) the inner maximization problem can be difficult to solve, (they) it
			is possible to obtain sufficiently good solutions using randomly re-started projected gradient descent 
			\item (Previous works) one-step adversaries, (they) multi-step methods
			\item (Previous works)  FGSM --FGSM-only evaluations are not fully reliable, (they) PGD
		\end{itemize}
	}
\end{frame}

\section{references}
\begin{frame}[allowframebreaks]
  {\tiny
    \frametitle{References}
    \bibliographystyle{ieeetr}
    \bibliography{bib/references.bib}
  }
\end{frame}


\end{document}
