\section{Adversarial Training}

\transitionFrame{Adversarially Robust Training}

\begin{frame}{Solving the Outer Minimization}
  \onslide<+->{%
    \begin{equation}\label{eq:MinimaxTrainRepeat}
      \green{\min_{\params} \rho(\params)} \text{, where } \rho(\params) = \mathbb{E}_{(\X,\y) \sim \distr} \sbrack{\red{\max_{\delta \in \sPerturb} \loss (\X + \perturb, \y ; \params)}}
    \end{equation}
  }

  \begin{itemize}[<+->]
    \item We talked about solving the \red{inner maximization} to create the adversarial examples.

    \vspace{13pt}
    \item \textbf{Question}: What do we need to solve to train adversarially robust networks?
    \vspace{5pt}
    \item \textbf{Answer}: Solve the \green{outer minimization}

    \vspace{13pt}
    \item \textbf{Question}: What algorithm can use to solve the outer maximization?
    \vspace{5pt}
    \item \textbf{Answer}: Stochastic gradient descent (SGD) on the adversarial examples
      \begin{itemize}[<+->]
        \setlength{\itemsep}{6pt}
        \item \textit{Intuition}: SGD on adv.\ examples' loss reduces the \red{inner maximization}
        \item \textit{Takeaway}: Given an algorithm that transforms training examples into adv.\ examples (e.g.,~PGD), the rest of the \textbf{\blue{training process proceeds normally}}
      \end{itemize}
  \end{itemize}
\end{frame}


\begin{frame}{Why Should I Believe the Preceding Claims are True?}
  \onslide<+->{}
  \onslide<+->{\textbf{Answer}: You shouldn't.  The preceding explanation is definitely \red{not a proof}.}

  \vspace{20pt}
  \onslide<+->{\madry\ rely on \textbf{\blue{Danskin's Theorem}} that states gradients at inner maximizers correspond to descent directions for the complete problem}

  \vspace{20pt}
  \onslide<+->{\textbf{Problem}: Multiple assumptions made by Danskin's Theorem's \red{do not apply} here, e.g.,~continuously differentiable function, only approximate inner maximizers etc.}
  \begin{itemize}[<+->]
    \setlength{\itemsep}{6pt}
    \item Empirical results show that despite Danskin's not holding the algorithm is addressing the issues
    \item A more complete discussion of how Danksin's applies to this problem is in Appendix~A (see Arxiv version) and is beyond the scope of this talk
  \end{itemize}
\end{frame}
